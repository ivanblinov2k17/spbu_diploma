
% ОБЯЗАТЕЛЬНО ИМЕННО ТАКОЙ documentclass!
% (Основной кегль = 14pt, поэтому необходим extsizes)
% Формат, разумеется, А4
% article потому что стандарт не подразумевает разделов
% Глава = section, Параграф = subsection
% (понятия "глава" и "параграф" из документа, описывающего диплом)
\documentclass[a4paper,article,14pt]{extarticle}

% Подключаем главный пакет со всем необходимым
\usepackage{spbudiploma_tempora}

% Пакеты по желанию (самые распространенные)
% Хитрые мат. символы
\usepackage{euscript}
% Таблицы
\usepackage{longtable}
\usepackage{makecell}
% Картинки (можно встявлять даже pdf)
\usepackage[pdftex]{graphicx}
\graphicspath{{figure_1/}{rgbcolor/}}

\usepackage{amsthm,amssymb, amsmath}
% remove leading space of mod
\usepackage{textcomp}


\newcommand{\Mod}[1]{\ \mathrm{mod}\ #1}

\begin{document}

% Титульник в файле titlepage.tex
\input{titlepage.tex}

% Содержание
\tableofcontents
\pagebreak

\specialsection{Введение}
В настоящее время JavaScript является одним из самых популярных языков программирования. Он широко используется для создания приложений,
исполняемых и со стороны клиента в браузере, и со стороны сервера. Так же распространены нативные приложения для мобильных устройств, 
Progressive Web Apps - гибриды нативных приложений и сайтов. В первую очередь это связано с развитием интернета и увеличением объема
передаваемых по сети данных. Вместе с этим растет потребность в безопасности данных, которые представляют собой некоторую ценность.
Потребность передачи секретных данных возникает у ученых, военных, в судопроизводстве, бизнесе. 
Традиционные методы защиты информации предоставляет криптография. Чаще всего информация защищается с помощью секретного алгоритма или ключа.
Но у такого подхода есть проблемы: если злоумышленник перехватит ключ или скомпрометирует одну из сторон, то он легко получит доступ к секрету.
Также, при необходимости разделить секретную информацию между какой-то группой людей приходится устанавливать соединения между каждой парой из группы,
что негативно сказывается на безопасности секрета.

В 1979 году A. Shamir представил (ссылка) алгоритм 
разделения секрета, который позволяет разбить секрет на $n$ долей таким образом, что знание $K$ и более долей позволяет восстановить 
секрет, а знание $K-1$ и менее долей делает восстановление секрета невозможным. В последние десятилетия было предложено множество 
алгоритмов разделения секрета для электронных изображений. В данной работе будет рассмотрен и дополнен алгоритм обратимого 
разделения секрета, реализована библиотека для использования в веб-приложениях и пример минимального проекта, использующего эту 
библиотеку.

\newpage
\specialsection{Цель и постановка задачи}

Целью данной работы является написание библиотеки для языка JavaScript, для разделения секретного цветного электронного изображения,
с долями, не подобными шуму. Для достижения этой цели были поставлены следующие задачи:
\begin{enumerate}
    \item Исследование предметной области
    \item Выбор алгоритма
    \item Модификация алгоритма для работы с цветным секретным изображением
    \item Написание библиотеки
    \item Написание минимального веб-приложения, позволяющего продемонстрировать работу программы
    \item Тестирование библиотеки и сравнение с имплементациями на других языках 
\end{enumerate}

\newpage
\specialsection{Обзор литературы}

В рамках спецификации современных стандартов, базовые сценарии поведения пользователей призваны к ответу. Банальные, но неопровержимые выводы, а также представители современных социальных резервов формируют глобальную экономическую сеть и при этом - представлены в исключительно положительном свете.

Есть над чем задуматься: предприниматели в сети интернет будут описаны максимально подробно. Приятно, граждане, наблюдать, как сторонники тоталитаризма в науке заблокированы в рамках своих собственных рациональных ограничений. Есть над чем задуматься: некоторые особенности внутренней политики объявлены нарушающими общечеловеческие нормы этики и морали. Как принято считать, тщательные исследования конкурентов смешаны с неуникальными данными до степени совершенной неузнаваемости, из-за чего возрастает их статус бесполезности.

Лишь предприниматели в сети интернет, которые представляют собой яркий пример континентально-европейского типа политической культуры, будут преданы социально-демократической анафеме. Есть над чем задуматься: стремящиеся вытеснить традиционное производство, нанотехнологии являются только методом политического участия и ограничены исключительно образом мышления! Разнообразный и богатый опыт говорит нам, что постоянный количественный рост и сфера нашей активности напрямую зависит от новых предложений.

\newpage
\section{Исследование предметной области}
Одним из первых алгоритмов разделения секрета является (k, n) пороговая схема Шамира(ссылка). В ее основе лежит интерполяция 
полиномов. Пусть $D$ -- некоторая секретная информация, представленная в форме числа. Выберем простое число $p: p > D, p > N$.
Чтобы разделить секрет на $n$ частей возьмем случайный полином степени $k-1$ 
\begin{equation}
    q(x) = a_0 + a_1 x +...+ a_{k-1} x^{k-1},
    a_0=D, a_i<p
\end{equation}
и вычислим
\begin{equation}
    D_1=q(1)\Mod{p}, ..., D_i=q(i)\Mod{p}, ..., D_n=q(n)\Mod{p}
\end{equation}
Число $p$ будет публичным для всех участников, числа $D_i$ назовем долями. Участника схемы, который хочет разделить секрет и 
формирует доли назовем дилером.

Имея $k$ и более долей можно восстановить секрет $D$ при помощи полиномиальной интерполяции.
%добавить инфу про восстановление 
Допустим, злоумышленнику удалось получить доступ к $k-1$ долям, тогда для каждого $D': 0<D'<p$ он может восстановить единственный полином степени $k-1$, такой, что $q_0=D'$ и 
$q_i=D_i$. Так как $a_i$ случайны, эти $p$ полиномов с одинаковой вероятностью являются искомыми, злоумышленник не получает никакой
информации о секрете. 

Схема Шамира позволяет разделить секрет, представленный в форме числа и используется в основном для защиты ключей. Изображение так же 
можно представить в форме числа, но при обычном размере изображения (для примера 256х256) и значениях пикселя (0-255)х3 для rgb изображений
будет тратиться огромное количество памяти. Поэтому на основе схемы Шамира были разработаны алгоритмы разделения секрета для изображений. 
Их можно разделить на три категории - схемы визуальной криптографии(VCS), полиномиальные схемы и схемы, основанные на Китайской теоремы об остатках. 

В 1994 году Moni Naor и Adi Shamir (ссылка) представили первую VCS, на ее основе были разработаны другие модификации.
В VCS схемах доли обычно печатаются на прозрачных носителях и секрет восстанавливается путем наложения частей друг на друга. 
Основным преимуществом таких схем является отсутствие необходимости вычислений при восстановлении секрета.
Примечательной для применения в веб-разработке является VCS схема WEB-VC (ссылка). Алгоритм восстановления секрета в ней основан 
на возможности установить прозрачность элемента в таблице каскадных стилей (CSS). Основными недостатками таких схем является 
наличие помех в восстановленном секретном изображении и возможность использования только с бинарными изображениями.

Полиномиальные схемы используются чаще из-за лучшего качества восстановленного секрета и в общем случае 
не требуют увеличения количества пикселей. Но у них есть и недостатки -- относительно высокая вычислительная сложность 
восстановления секрета $O(k \times log^2(k))$ для каждого пикселя и небольшие потери в качестве восстановленного секретного изображения.
% Дополнить про 1 достойную схему

Схемы, основанные на Китайской теореме об остатках имеют более низкую сложность операции восстановления $O(k)$ и позволяют 
восстановить секрет без потерь. Недостатком таких схем является ограниченное количество участников(ссылка). Таким образом, 
эти схемы отлично подходят для устройств с низкой вычислительной мощностью и для испозьзования в веб-приложениях.

Во многих схемах дилер отправляет участникам шумо-подобные доли. Введём понятие изображений для прикрытия -- это произвольные 
изображения, использующиеся для генерации долей. Сгенерированные доли являются изображениями, похожими на изображения прикрытия. 
Использование изображений прикрытия вместо шумо-подобных долей снижает риск привлечения внимания к долям злоумышленников, улучшает возможности 
по их менеджменту. 

В данной работе будет рассматриваться алгоритм Reversible Image Secret Sharing (ссылка). Он основан на китайской 
теореме об остатках. В качестве секретной картинки выступает изображение в оттенках серого $(0-255)$, картинками для прикрытия являются
бинарные изображения.


\newpage
\section{Описание алгоритма}
Начнем описание работы алгоритма с формулировки Китайской теоремы об остатках.

Если $ a_1,...,a_n \in N $ попарно взаимно просты, то для 
$$\forall r_1,...,r_n \in N : 0\leq r_i<a_i, \forall i\in \overline{1,n}$$
найдется $N: N \Mod a_i = p_i, \forall i\in \overline{1,n}$ 

Эта теорема позволяет за линейное время решать систему линейных модулярных уравнений следующего типа:
\begin{gather}
    \begin{cases}
    y \equiv a_1 \Mod m_1 \\
    y \equiv a_2 \Mod m_2 \\
    ... \\
    y \equiv a_k \Mod m_k
    \end{cases}
\end{gather}
Алгоритм решения (ссылка):
\begin{enumerate}
    \item Вычисляем $M=\prod\limits_{i = 1}^k m_i$
    \item $\forall i\in \overline{1,k}$ вычисляем $M_i={{M} \over {m_i}}$
    \item С помощью расширенного алгоритма Евклида $\forall i\in \overline{1,k}$ находим ${M_i}^{-1}$ обратное по модулю для $M_i$
    \item Получаем $y \equiv \sum\limits_{i=1}^k a_i M_i {M_i}^{-1} \Mod M$
\end{enumerate}

Предложенный алгоритм состоит из двух частей: формирование долей и восстановление секрета. Опишем их более подробно.

\subsection{Формирование долей}
Описание входных данных:

\begin{itemize}
    \setlength{\itemindent}{3em}
    \item Секретное изображение $S$ размера $W_S \times H_S$ пикселей в оттенках серого (значения пикселей 0-255)
    \item $n$ - количество долей
    \item $k$ - минимальное количество долей для восстановления секрета
    \item $n$ изображений $C_i$ размера $W_S \times H_S$ - бинарные (значения пикселей 0-1) изображения прикрытия для каждого из участников
\end{itemize}

Описание выходных данных:

\begin{itemize}
    \setlength{\itemindent}{3em}
    \item $n$ изображений $SC_i$ размера $W_S \times H_S$ - сгенерированные доли
    \item $m_i$ - приватное число для каждой доли
    \item $p, T$ - публичные для всех участников числа для восстановления секрета
\end{itemize}

Алгоритм:
\begin{enumerate}
    \setlength{\itemindent}{3em}
    \item Выберем число $p$ и $n$ взаимно простых чисел $m_i$ таких, что 
    $$128 \leq p < m_i \leq 256, \text{НОД}(m_i, p)=1, \forall i \in \overline{1,n}$$ 
    \item Вычислим $M=\prod\limits_{i = 1}^k m_i$, $N=\prod\limits_{i = 1}^{k-1} m_{n-i+1}$
    \item Если $M<pN$ перейдем к шагу 1
    \item Вычислим $T=\left[ {{\left\lfloor{{M}\over{p}}-1\right\rfloor}\over{2}} \right]$
    \item Для каждого секретного пикселя $x$ с координатами $[w, h]$ повторяем шаги 6-7
    \item Если $0 \leq x < p$, выберем случайное $ A \in \left[T+1, {\left\lfloor{{M} \over {p}} - 1\right\rfloor}\right]$ и вычислим
    $y = x + Ap$. 
    
    Если $x \geq p$ выберем случайное $ A \in [0, T)$ и вычислим $y = x - p + Ap$
    \item Если выполняется одно из следующих условий, то вычисляем $a_i = y \Mod p$, устанавливаем $SC_i=a_i$ и
    переходим к следующему пикселю, иначе возвращаемся на шаг 6.
    \begin{gather}
        \begin{cases}
        SC_i[w,h] \geq TH_{i1}, \text{ если } C_i[w,h] = 1 \\
        SC_i[w,h] \leq TH_{i0}, \text{ если } C_i[w,h] = 0
        \end{cases}
    \end{gather}
    \begin{equation}
        \text{при } TH_{i0} = {{m_i} \over 2} - TH,\text{ } TH_{i1} = {{m_i} \over 2} + TH
    \end{equation}
\end{enumerate}

\subsection{Восстановление секрета}

Описание входных данных:
\begin{itemize}
    \setlength{\itemindent}{3em}
    \item $n$ долей $SC_i$ ($n \geq k$) и соответствующие им $m_i$
    \item Публичные числа $T, p$
\end{itemize}

Описаные выходных данных:
\begin{itemize}
    \setlength{\itemindent}{3em}
    \item Восстановленный секрет $S'$
    \item Восстановленные изображения прикрытия $C_{i}'$ размера $[W_S, H_S]$
\end{itemize}

Алгоритм
\begin{enumerate}
    \setlength{\itemindent}{3em}
    \item Восстановливаем изображения прикрытия с помощью бинаризации. Для каждого пикселя $C_{i}'[w, h]$ устанавливаем значение 
    $$\text{Если } SC_{i}[w, h]>{{m_i}\over{2}} \text{, то } 1 \text{, иначе } 0$$
    \item Для каждой позиции пикселя $[w, h]$, $a_i = SC_i[w, h]$, с помощью описанного выше алгоритма (ссылка) решаем систему 
    линейных уравнений по модулю:
    \begin{gather}
        \begin{cases}
        y \equiv a_1 \Mod m_1 \\
        y \equiv a_2 \Mod m_2 \\
        ... \\
        y \equiv a_n \Mod m_n
        \end{cases}
    \end{gather}
    \item Вычисляем $T^{*}=\left\lfloor{{y} \over {p}}\right\rfloor$. Если $T^{*}\leq T$, то $x = y \Mod p$, иначе $x = (y \Mod p) + p$.
    
    $S'[w, h] = x$
\end{enumerate}

\subsection{Комментарии к алгоритму}
\begin{enumerate}
    \item Число $p$ в алгоритме 1 (ссылка) выбирается наименьшим из возможных, а числа $m_i$ выбираются наибольшими для достижения 
    большего диапазона распределения значения пикселя в долях.

    \item Параметр $TH$ имеет весомую роль в качестве сгенерированных долей, времени генерации и безопасности. Этот параметр устанавливается 
    дилером и влияет на $N_A$ -- число возможных значений А в шаге 6 алгоритма 1.
    В общем случае, $N_A = T$. Для того, чтобы удовлетворять условию на шаге 7 алгоритма, значение $N_A$ уменьшается до 
    $N_A = T \times \prod\limits_{i = 1}^n \left({1\over 2} \times {TH_{i0}\over m_i} + {1\over 2} \times {{m_i - TH_{i0}}\over m_i}\right)$.
    
    При увеличении $TH$ уменьшается $N_A$, увеличивается качество сгенерированных долей и время на генерацию.
    При увеличении $N_A$ улучшается безопасность, так как количество значений для перебора равняется $T^{N_A}$. 
    Требуется, чтобы $N_A \geq 2$, так как при $N_A = 1$ в шаге 6 алгоритма будет 
    повторно использоваться одно и тоже значение $A$, что приводит к проблемам с безопасностью. Экстремальной точкой для качества долей 
    является $TH = 112$. Экспериментальные данные можно увидеть на ((Рисунке 1)). 

    \item Качество картинки по сравнению с изначальной будем измерять с помощью пикового отношения сигнала к шуму -- $PSNR$. Эту метрику чаще всего 
    определяют с помощью среднеквадратичной ошибки $MSE$. Пусть $I$ -- исходное изображение размера $m \times n$, $K$ -- зашумленная версия $I$. Тогда 
    \begin{equation}
        MSE = {1\over mn} \sum_{i=0}^{m-1} \sum_{j=0}^{n-1}{|I[i, j] - K[i, j]|}^2
    \end{equation}
    \begin{equation}
        PSNR = 10 \log_{10}\left({{MAX_i^2} \over {MSE}} \right)
    \end{equation}

    \item Мощностью встраивания $EC$ (embedding capacity) называется отношение количества бит информации, встраиваемых в изображение, к размеру изображения 
    и определяется формулой:
    $$EC = N\over{H \times W}$$
    $N$ -- число бит секрета, $H \times W$ -- размер изображения.
    
    Для данного алгоритма $EC$ примет следующий вид:
    $$EC = $$

    \item Результаты работы алгоритма для $n=5$, $k=4$, $TH=16$ показаны на Рисунке \ref{fig:experimental_grey_images} 
\end{enumerate}
\begin{figure}[ph!]
    \begin{minipage}[h]{0.3\linewidth}
        \center{\includegraphics[width=1\linewidth]{lena.png} \\ $S$ }
    \end{minipage}
    \begin{minipage}[h]{0.3\linewidth}
        \center{\includegraphics[width=1\linewidth]{share1.png} \\ $SC_1$ }
    \end{minipage}
    \begin{minipage}[h]{0.3\linewidth}
        \center{\includegraphics[width=1\linewidth]{share2.png} \\ $SC_2$}
    \end{minipage}
    \begin{minipage}[h]{0.3\linewidth}
        \center{\includegraphics[width=1\linewidth]{share3.png} \\ $SC_3$}
    \end{minipage}
    \begin{minipage}[h]{0.3\linewidth}
        \center{\includegraphics[width=1\linewidth]{share4.png} \\ $SC_4$}
    \end{minipage}
    \begin{minipage}[h]{0.3\linewidth}
        \center{\includegraphics[width=1\linewidth]{share5.png} \\ $SC_5$}
    \end{minipage}
    \begin{minipage}[h]{0.3\linewidth}
        \center{\includegraphics[width=1\linewidth]{binary1.png} \\ $C_{1}'=C_{1}$ \\ $PSNR=\infty$}
    \end{minipage}
    \begin{minipage}[h]{0.3\linewidth}
        \center{\includegraphics[width=1\linewidth]{binary2.png} \\ $C_{2}'=C_{2}$ \\ $PSNR=\infty$}
    \end{minipage}
    \begin{minipage}[h]{0.3\linewidth}
        \center{\includegraphics[width=1\linewidth]{binary3.png} \\ $C_{3}'=C_{3}$ \\ $PSNR=\infty$}
    \end{minipage}
    \begin{minipage}[h]{0.3\linewidth}
        \center{\includegraphics[width=1\linewidth]{binary4.png} \\ $C_{4}'=C_{4}$ \\ $PSNR=\infty$}
    \end{minipage}
    \hfill
    \begin{minipage}[h]{0.3\linewidth}
        \center{\includegraphics[width=1\linewidth]{binary5.png} \\ $C_{5}'=C_{5}$ \\ $PSNR=\infty$}
    \end{minipage}
    \hfill
    \begin{minipage}[h]{0.3\linewidth}
        \center{\includegraphics[width=1\linewidth]{lena.png} \\ $S_{1234}'=S_{2345}'=S$ \\ $PSNR=\infty$ }
    \end{minipage}
    \caption{Экспериментальные изображения}
    %задает метку рисунка чтобы после ссылаться
    \label{fig:experimental_grey_images}
\end{figure}

\newpage
\section{Модификация алгоритма}

Описанный выше алгоритм отлично подходит для цели работы, за исключением цвета картинки. Поэтому
было принято решение расширить исходный алгоритм для использования с цветными секретными картинками. 
Это было достигнуто с помощью увеличения количества пикселей в картинках прикрытия и кодирования каждого канала цвета 
в определенном пикселе доли. На рисунке \ref{fig:channels} представлена схема расположения каналов в доле. 

Размер исходной цветной картинки -- $2\times 1$, размер доли -- $8 \times 1$.
Белый сегмент при использовании с картинками формата rgba отвечает за кодирование канала прозрачности alpha,
для rgb не принимает участия в разделении секрета и в зависимости от значения картинки прикрытия $C_i = 0 | 1$ принимает значения $0 | 255$.

\begin{figure}[h]
    \center{\includegraphics[width=1\linewidth]{rgba.png}}
    \caption{Схема расположения каналов}
    \label{fig:channels}
\end{figure}

Для того, чтобы избежать существенного увеличения размера передаваемого изображения, доли сохраняются в формате $PGM$ -- portable gray map. В нем для 
кодирования каждого пикселя используется 8 бит. Таким образом, изображение доли будет занимать на диске такое же пространство, как и секретное rgba изображение.


 
Описание модификаций в алгоритме

рассказ про PSNR, EC

\newpage
\section{Имплементация библиотеки}

раасказ про npm модули
рассказ про библиотеку и как я ее офигенно загрузил на нпм и какая она в открытом доступе
пару слов про приложение со скринами

\newpage
\section{Написание сайта, использующего библиотеку}

фейковые замеры производительности, скрины, про то как я картинки в серые превращал, какие библиотеки использовал помимо



Нумерованная формула:

\begin{equation}
    i^2 = -1.
    \label{eq:my_ref}
\end{equation}

Тест ссылки на формулу \ref{eq:my_ref}.

\newpage
\specialsection{Выводы}
Жизнь --- тлен.
\pagebreak

\specialsection{Заключение}

С другой стороны, консультация с широким активом обеспечивает актуальность форм воздействия. Следует отметить, что выбранный нами инновационный путь создает необходимость включения в производственный план целого ряда внеочередных мероприятий с учетом комплекса благоприятных перспектив. В частности, реализация намеченных плановых заданий влечет за собой процесс внедрения и модернизации поэтапного и последовательного развития общества. В частности, новая модель организационной деятельности способствует подготовке и реализации стандартных подходов и тому подобных экспериментов.

\newpage    
% Библиография в cpsconf стиле
% Аргумент {1} ниже включает переопределенный стиль с выравниванием слева
\begin{thebibliography}{1}
\bibitem{voc} Griffin D.W., Lim J.S. \flqq Multiband excitation vocoder\frqq. IEEE ASSP-36 (8), 1988, pp. 1223-1235.
\bibitem{vo2} Griffin D.W., Lim J.S. \flqq Multiband excitation vocoder\frqq. IEEE ASSP-36 (8), 1988, pp. 1223-1235.
\end{thebibliography}
\end{document}