% ОБЯЗАТЕЛЬНО ИМЕННО ТАКОЙ documentclass!
% (Основной кегль = 14pt, поэтому необходим extsizes)
% Формат, разумеется, А4
% article потому что стандарт не подразумевает разделов
% Глава = section, Параграф = subsection
% (понятия "глава" и "параграф" из документа, описывающего диплом)
\documentclass[a4paper,article,14pt]{extarticle}

% Подключаем главный пакет со всем необходимым
\usepackage{spbudiploma_tempora}

% Пакеты по желанию (самые распространенные)
% Хитрые мат. символы
\usepackage{euscript}
% Таблицы
\usepackage{longtable}
\usepackage{makecell}
% Картинки (можно встявлять даже pdf)
\usepackage[pdftex]{graphicx}

\usepackage{amsthm,amssymb, amsmath}
\usepackage{textcomp}


\begin{document}

% Титульник в файле titlepage.tex
\input{titlepage.tex}

% Содержание
\tableofcontents
\pagebreak

\specialsection{Введение}

С развитием современных медиа и интернета увеличивается объем передаваемых данных. Вместе с этим растет потребность в безопасности 
данных, которые представляют собой некоторую ценность. Традиционные методы защиты информации представляет криптография. Чаще всего 
информация защищается с помощью секретного алгоритма или ключа. Но у такого подхода есть проблемы: если злоумышленник перехватит 
ключ или скомпрометирует одну из сторон, то он легко получит доступ к секрету.

В 1979 году A. Shamir представил (ссылка) алгоритм 
разделения секрета, который позволяет разбить секрет на $n$ долей таким образом, что знание $K$ и более долей позволяет восстановить 
секрет, а знание $K-1$ и менее долей делает восстановление секрета невозможным. В последние десятилетия было предложено множество 
алгоритмов разделения секрета для электронных изображений. В данной работе будет рассмотрен и дополнен алгоритм обратимого 
разделения секрета, реализована библиотека для использования в веб-приложениях и пример минимального проекта, использующего эту 
библиотеку

\newpage
\specialsection{Цель и постановка задачи}

Целью данной работы является написание библиотеки для языка JavaScript, для разделения секретного цветного электронного изображения,
с долями, не подобными шуму. Для достижения этой цели были поставлены следующие задачи:
\begin{enumerate}
    \item Исследование предметной области
    \item Выбор алгоритма
    \item Модификация алгоритма для соответствия поставленным требованиям
    \item Написание библиотеки
    \item Написание минимального веб-приложения, позволяющего продемонстрировать работу программы
    \item Тестирование библиотеки и сравнение с имплементациями на других языках 
\end{enumerate}

\newpage
\specialsection{Обзор литературы}

В рамках спецификации современных стандартов, базовые сценарии поведения пользователей призваны к ответу. Банальные, но неопровержимые выводы, а также представители современных социальных резервов формируют глобальную экономическую сеть и при этом - представлены в исключительно положительном свете.

Есть над чем задуматься: предприниматели в сети интернет будут описаны максимально подробно. Приятно, граждане, наблюдать, как сторонники тоталитаризма в науке заблокированы в рамках своих собственных рациональных ограничений. Есть над чем задуматься: некоторые особенности внутренней политики объявлены нарушающими общечеловеческие нормы этики и морали. Как принято считать, тщательные исследования конкурентов смешаны с неуникальными данными до степени совершенной неузнаваемости, из-за чего возрастает их статус бесполезности.

Лишь предприниматели в сети интернет, которые представляют собой яркий пример континентально-европейского типа политической культуры, будут преданы социально-демократической анафеме. Есть над чем задуматься: стремящиеся вытеснить традиционное производство, нанотехнологии являются только методом политического участия и ограничены исключительно образом мышления! Разнообразный и богатый опыт говорит нам, что постоянный количественный рост и сфера нашей активности напрямую зависит от новых предложений.

\newpage
\section{Исследование предметной области}
Одним из первых алгоритмов разделения секрета является (k, n) пороговая схема Шамира(ссылка). В ее основе лежит интерполяция 
полиномов. Пусть $D$ -- некоторая секретная информация, представленная в форме числа. Выберем простое число $p: p > D, p > N$.
Чтобы разделить секрет на $n$ частей возьмем случайный полином степени $k-1$ 
\begin{equation}
    q(x) = a_0 + a_1 x +...+ a_{k-1} x^{k-1},
    a_0=D, a_i<p
\end{equation}
и вычислим
\begin{equation}
    D_1=q(1)\mod p, ..., D_i=q(i)\mod p, ..., D_n=q(n)\mod p
\end{equation}
Число $p$ будет публичным для всех участников, числа $D_i$ назовем долями.

Имея $k$ и более долей можно восстановить секрет $D$ при помощи полиномиальной интерполяции. Допустим злоумышленнику удалось получить
доступ к $k-1$ долям, тогда для каждого $D': 0<D'<p$ он может восстановить единственный полином степени $k-1$, такой, что $q_0=D$ и 
$q_i=D_i$. Но так как по определению эти $p$ полиномов с одинаковой вероятностью являются искомыми, злоумышленник не получает никакой
информации о секрете.

На основе схемы Шамира были разработаны алгоритмы разделения секрета для изображений. Их можно разделить на три категории - схемы 
визуальной криптографии(VCS), полиномиальные и схемы, основанные на Китайской теоремы об остатках. 

В VCS схемах изображение обычно печатается на прозрачных носителях и восстанавливается путем наложения частей друг на друга. Такие 
схемы обычно характеризуются плохим качеством изображения и значительным увеличением количества пикселей в долях. Их плюсом является
отсутствие необходимости вычислений при восстановлении секрета. (ремарка про работу CSS и ссылка )

Полиномиальные схемы используются чаще из-за лучшего качества восстановленного секрета и не требуют увеличения количества пикселей.
Но у них есть и недостатки - относительно высокая вычислительная сложность восстановления секрета $O(k*log^2(k))$ и небольшие потери
в качестве изображения.

В данной работе будет рассматриваться алгоритм за авторством Xuehu Yan, Yuliang Lu, Lintao Liu. Он основан на китайской теореме об 
остатках. В качестве секретной картинки выступает изображение в оттенках серого $(0-255)$.
Так же вводится понятие изображений для прикрытия -- это изображения, использующиеся для генерации долей. Сгенерированные доли 
являются изображениями в оттенках серого, похожими на изображения прикрытия. Использование изображений прикрытия 
вместо шумо-подобных долей снижает риск привлечения внимания к долям злоумышленников, улучшает возможности 
по их менеджменту, позволяет за линейное время восстановить исходную бинарную картинку при надобности. 

формулировка китайской теоремы
про то, как решать линейное уравнение 

формулировка алгоритма (2-3) стран
ремарка про коэф TH
Так как использование черно-белых картинок является не самым частым сценарием в веб-разработке, в ходе работы было принято решение 
расширить исходный алгоритм для использования с цветными секретными картинками. Это было достигнуто с помощью увеличения количества
пикселей в картинках прикрытия и кодирования каждого канала цвета в определенном пикселе доли.
формулировка улучшения для цветных картинок

рассказ про библиотеку и как я ее офигенно загрузил на нпм и какая она в открытом доступе
пару слов про приложение со скринами




Нумерованная формула:

\begin{equation}
    i^2 = -1.
    \label{eq:my_ref}
\end{equation}

Тест ссылки на формулу \ref{eq:my_ref}.


\specialsection{Выводы}
Жизнь --- тлен.
\pagebreak

\specialsection{Заключение}

С другой стороны, консультация с широким активом обеспечивает актуальность форм воздействия. Следует отметить, что выбранный нами инновационный путь создает необходимость включения в производственный план целого ряда внеочередных мероприятий с учетом комплекса благоприятных перспектив. В частности, реализация намеченных плановых заданий влечет за собой процесс внедрения и модернизации поэтапного и последовательного развития общества. В частности, новая модель организационной деятельности способствует подготовке и реализации стандартных подходов и тому подобных экспериментов.

% Библиография в cpsconf стиле
% Аргумент {1} ниже включает переопределенный стиль с выравниванием слева
\begin{thebibliography}{1}
\bibitem{voc} Griffin D.W., Lim J.S. \flqq Multiband excitation vocoder\frqq. IEEE ASSP-36 (8), 1988, pp. 1223-1235.
\bibitem{vo2} Griffin D.W., Lim J.S. \flqq Multiband excitation vocoder\frqq. IEEE ASSP-36 (8), 1988, pp. 1223-1235.
\end{thebibliography}
\end{document}